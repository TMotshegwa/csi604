% !TEX TS-program = pdflatex
% !TEX encoding = UTF-8 Unicode

% This is a simple template for a LaTeX document using the "article" class.
% See "book", "report", "letter" for other types of document.

\documentclass[11pt]{article} % use larger type; default would be 10pt

\usepackage[utf8]{inputenc} % set input encoding (not needed with XeLaTeX)

%%% Examples of Article customizations
% These packages are optional, depending whether you want the features they provide.
% See the LaTeX Companion or other references for full information.

%%% PAGE DIMENSIONS
\usepackage{geometry} % to change the page dimensions
\geometry{a4paper} % or letterpaper (US) or a5paper or....
% \geometry{margin=2in} % for example, change the margins to 2 inches all round
% \geometry{landscape} % set up the page for landscape
%   read geometry.pdf for detailed page layout information

\usepackage{graphicx} % support the \includegraphics command and options

% \usepackage[parfill]{parskip} % Activate to begin paragraphs with an empty line rather than an indent

%%% PACKAGES
\usepackage{booktabs} % for much better looking tables
\usepackage{array} % for better arrays (eg matrices) in maths
\usepackage{paralist} % very flexible & customisable lists (eg. enumerate/itemize, etc.)
\usepackage{verbatim} % adds environment for commenting out blocks of text & for better verbatim
\usepackage{subfig} % make it possible to include more than one captioned figure/table in a single float
% These packages are all incorporated in the memoir class to one degree or another...

%%% HEADERS & FOOTERS
\usepackage{fancyhdr} % This should be set AFTER setting up the page geometry
\pagestyle{fancy} % options: empty , plain , fancy
\renewcommand{\headrulewidth}{0pt} % customise the layout...
\lhead{}\chead{}\rhead{}
\lfoot{}\cfoot{\thepage}\rfoot{}

%%% SECTION TITLE APPEARANCE
\usepackage{sectsty}
\allsectionsfont{\sffamily\mdseries\upshape} % (See the fntguide.pdf for font help)
% (This matches ConTeXt defaults)

%%% ToC (table of contents) APPEARANCE
\usepackage[nottoc,notlof,notlot]{tocbibind} % Put the bibliography in the ToC
\usepackage[titles,subfigure]{tocloft} % Alter the style of the Table of Contents
\renewcommand{\cftsecfont}{\rmfamily\mdseries\upshape}
\renewcommand{\cftsecpagefont}{\rmfamily\mdseries\upshape} % No bold!

%%% END Article customizations

%%% The "real" document content comes below...

\title{ {\Huge \textbf{University Of Botswana}} \\ Department of Computer Science \\ 
\vspace{1.0cm}CSI604 ADVANCED OPERATING SYSTEMS HOMEWORK 1 \\
Survey - Operating Systems Design, Implementation and Applications \\ 
DEADLINE  26th September 2016, 23:59
}
\author{Instructor: Dr Tshiamo Motshegwa}
%\date{} % Activate to display a given date or no date (if empty),
         % otherwise the current date is printed 

\begin{document}
\maketitle

\section{Overview}


\section{Instructions}

\noindent Conduct  a short literature review on the topic of \textit{Operating Systems Design, implementation and Applications}. Identify relevant journal papers and create a bibliography, e.g Bibtex (or whatever you prefer) UB online library resources and science direct (http://www.sciencedirect.com/), IEEExplore (http://ieeexplore.ieee.org/), citeseerx (http://citeseerx.ist.psu.edu/) etc. are a good start.\\ 

\noindent Key standard operating systems textbooks like \cite{osDesignImpTanenbaum1997}, \cite{osInternalsAndDesignStallings1996} also offer further reading on key publications on operating systems internals, design principles and implementation considering case studies like MINIX, Solaris, SVR4 UNIX and Windows NT and other important operating systems. Advanced operating systems topics are covered in \cite{advancedOSSinghal194} and \cite{centDistOSNutt192} for example


\noindent This is a wide subject, your literature review does not have to cover all ground. It may be useful to consider some developments in the design and implemetation of an operating system from two views - as a extended machine/virtual machine and a resource manager considering some of the design and implementation issues in regards to the following
 
\begin{itemize}
\item Multiprogramming
\item Processes and Threads- e.g consider advanced concepts like Threading, Symmetric Multiprocessing, Microkernels etc
\item Memory Management and Virtual Memory
\item Concurrency
\item Scheduling
\item Security, protection and isolation
\end{itemize} 

\noindent The discussion can cover issues, solutions, algorithms and key publications on these. \\
 
\noindent It may also be interesting to look at design concepts like Object Orientation \cite{ooadBooch1994} and its benefit to the design of Operating Systems e.g better organisation of inherent complexity, reduced development effort, more extesible and maintanable systems. \\
\noindent Also with proliferation of networks \cite{dataCommsStallings1997}, clusters \cite{clustersPfister1997}, distributed systems \cite{princDSGarg1996}  - there is an active area of distributed operating systems,
considering Distributed Computing \cite{DCSCavasant1994} in managing complexity and presenting standard operating systems abstractions to users and applications, e.g distributed process managenment including process migration as discussed in \cite{procMigrationEskicioglu1990} and associated algorithms as discussed in  \cite{dosSingha1997} and \cite{dapRaynal1988} . There are a number of prototype distributed operating systems, and associated global and distributed file systems etc. There are a lot of publications and surveys on these e.g. \cite{Tanenbaum92theamoeba}, \cite{dosTanenbaum985}

\noindent There is also active area of research around application specific, embedded systems and operating systems considering contraints and realtime and other requirements for such platforms and applications, for example designing for multimedia, mobile devices,high perfomance computing, cloud



\noindent \paragraph{Example citing}. \\ 
\cite{osDesignImpTanenbaum1997} discusses Operating systems design and implementation seeking to provide a good balance between theory and practice in the teaching of operatng systems. \cite{Frohlich99highperformance} observes that the gap between object-oriented operating systems and  high performance parallel applications them originates from the complexity of assembling an operating system out of a complex collection of complex classes and proposes the EPOCH pproach to address this

\section{LEARNING OUTCOMES}
\noindent To conduct independent research on a given operating systems topic, looking at theory, practice, identifying key issues, solutions and discussion of the state of the art in the field and . 
\\ This exercise also help you acquire Typesetting, referencing and bibliograhy skills 

\section{SUBMISSION}
\begin{itemize}
\item Literature review document (No more than 10 pages) with citations and references - You can use Latex and Bibtex
\item Your Bibtex bibliography
\item 10 key papers you uncovered and referenced
\item  A one page poster
\end{itemize}

\section{SUBMSSION INSTRUCTIONS}

\noindent  Submit a zip file with all your files for the above at the gven moodle link

\section {TYPSETTING AND BIBLIOGRAPHY TOOLS}
\noindent Consider using the following tools - See links and on Moodle
\begin{enumerate}

\item Latex (you can Miktex - Latex on Windows)
\item Latex Front End/Editor, e.gTexWorks or an Eclipse plugin  (or can just use simple editors)
\item Bibtex and frontend bibliography reference manager e.g Jabref ( or can just use simple text editor to populate your bliography)
\item Example latex templates 
\end{enumerate}

\section{ASSESSMENT CRITERIA}

\noindent Marks will be awarded for the following

\begin{enumerate}
	\item Critial analysis 
	\item Document structure
	\item Bibliography
	\item Poster 
\end{enumerate}
%\section{TEXTBOOK AND REQUIRED MATERIALS}
%\begin{enumerate}
%\item \lbrack Dusseau \& Dusseau \rbrack \textit{Operating Systems: Three Easy Pieces}
% \item \lbrack K \& R\rbrack \textit{The C Programming Language, 2nd Edition}
% \item \lbrack Stevens\rbrack \textit{Advanced Programming in the UNIX Environment, 3rd Edition}
%\item \lbrack Stallings\rbrack \textit{Operating Systems: Internals and Design Principles, 8th Edition }
% \item \lbrack Tanenbaum\rbrack \textit{Modern Operating Systems (4th Edition) }
% \item \lbrack Anderson\rbrack \textit{Operating Systems: Principles and Practice, 2nd Edition}
%\item \lbrack Love\rbrack \textit{Linux Kernel Development (3rd Edition)}
%\item \lbrack Kirk McKusick\rbrack \textit{The Design and Implementation of the FreeBSD Operating System, 2nd Edition} 
%\item \lbrack Bach\rbrack \textit{The Design of the UNIX Operating System 1st Edition }
%\item \lbrack Bach\rbrack \textit{The Linux Programming Interface: A Linux and UNIX System Programming Handbook, 1st Edition}
%\item \lbrack Ben-Ari \rbrack \textit{Principles of Concurrent and Distributed Programming, 2nd Edition}
%\item \lbrack Silberschatz\rbrack \textit{Operating System Concepts, 8th Edition} 
%\end{enumerate}

\nocite{*} 							% Print all references regardless of whether they are cited in the document or not
%\bibliographystyle{plain} 					% Plain Referencing Style
\bibliographystyle{alpha} 					% Alpha Referencing Style
%\bibliographystyle{apalike} 				% Apalike Referencing Style
\bibliography{Operating-Systems-Bibliography} 	% Use your Biblograghy file e.g Operating-Systems-Bibliography.bib
\end{document}


